\documentclass{article}%
%Options -- Point size:  10pt (default), 11pt, 12pt
%        -- Paper size:  letterpaper (default), a4paper, a5paper, b5paper
%                        legalpaper, executivepaper
%        -- Orientation  (portrait is the default)
%                        landscape
%        -- Print size:  oneside (default), twoside
%        -- Quality      final(default), draft
%        -- Title page   notitlepage, titlepage(default)
%        -- Columns      onecolumn(default), twocolumn
%        -- Equation numbering (equation numbers on the right is the default)
%                        leqno
%        -- Displayed equations (centered is the default)
%                        fleqn (equations start at the same distance from the right side)
%        -- Open bibliography style (closed is the default)
%                        openbib
% For instance the command
%           \documentclass[a4paper,12pt,leqno]{article}
% ensures that the paper size is a4, the fonts are typeset at the size 12p
% and the equation numbers are on the left side
%
\usepackage{amsmath}%
\usepackage{amsfonts}%
\usepackage{amssymb}%
\usepackage{graphicx}


\begin{document}

\title{Open Rent}
\author{Project Plan}
\date{}
\maketitle

\section{Introduction}

This plan consist of the entire project plan to construct all the Group work home of the  discipline Software engineer proposed by Fernando Antonio De Araujo Chacon from computer Science Department of University of Brasília (UnB).

\section{Project organization}

See also https://github.com/flpinheiro/ProjetoES \\
This work is divided into the following content areas:\\
\begin{description}
	\item[Project Manager] Felipe Luís Pinheiro
	\item[Analyst] Wanderlan Alves de Jesus Brito
	\item[Architect] William Coelho da Silva
	\item[Tester] Andrey Calaça Resende
\end{description}

\section{Project practices and measurements}

The OpenUP component team will use OpenUP practices adapted to address the fact that we are doing content development rather than coding. Key artifacts include: Project defined process, project plan, iteration plan, tools, glossary, vision, system-wide requirements, usa-case model, use case, architecture notebook, user interface  project, database physical project, infrastructure, test cases.
Progress is tracked using two primary measurements using a point system. It is estimated that 1 point represents 2h of work:
\begin{itemize}
	\item Project backlog: The project backlog shows progress relative to overall work to be done within the project.
	\item Iteration backlog: The iteration backlog shows progress relative to work intended for the current iteration.
\end{itemize}

\section{Project milestones and objectives}


\begin{tabular}{|l|l|l|l|}
\hline
Iteration & Primary objectives            & milestone                & Target velocity \\ \hline
I1        & Objectives                    & 25/02/2021 to 04/03/2021 & 7               \\
          & 1.      Project Plan          &                          &                 \\
          & 2.      Iteration Plan 1      &                          &                 \\ \hline
I2        & Objectives                    & 05/03/2021 to 12/03/2021 & 7               \\
          & 1.      Iteration Plan 2      &                          &                 \\
          & 2.      Use-case Model        &                          &                 \\
          & 3.      Architecture Notebook &                          &                 \\
          & 4.      Smoke Test            &                          &                 \\
          & 5.      Glossary              &                          &                 \\ \hline
\end{tabular}


\section{Deployment}

\section{Lessons learned}

\end{document}
