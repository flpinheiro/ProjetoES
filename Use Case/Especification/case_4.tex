\section{descadastrar conta}

\subsection*{Brief Description}

Este caso descreve como os usuários podem descadastrar uma conta autenticada.
\subsection*{Actors}

\begin{itemize}
    \item \textbf{Interessados}
    \item \textbf{Proprietários}
\end{itemize}

\subsection*{Preconditions}
O usuário deve ter a conta autenticada no sistema.

\subsection*{Basic Flow of Events}

\begin{enumerate}
    \item  O usuário seleciona a opção referente a descadastrar sua conta.
    \item O sistema pede para que o usuário informe seu endereço de e-mail e sua senha para
    confirmar a identidade do usuário.
    \item O usuário informa os dados pedidos.
    \item O sistema remove a autenticação da conta do usuário.
    \item O sistema apaga os imóveis e as propostas cadastradas pelo usuário.
    \item O usuário é levado de volta para a página inicial.
\end{enumerate}

\subsection*{Alternative Flows}

\textbf{E-mail ou senha incorretos}

O sistema informa ao usuário que seus dados estão incorretos e não permite que a operação
de descadastro aconteça até que os dados sejam inseridos corretamente.

\textbf{Usuário não possui imóveis ou propostas de aluguel ativas}

O sistema informa que não existem imóveis ou propostas de aluguel para serem removidos, e
apenas remove a autenticação do usuário.

\subsection*{Post-conditions}

\textbf{Conta descadastrada com sucesso}

Se o usuário consegue descadastrar sua conta com sucesso o sistema o-redireciona para a
página inicial.

\textbf{Falha ao descadastrar conta}

Caso algum erro inesperado aconteça, o sistema reinicia a operação de descadastro de
conta.
\subsection*{Special Requirements} 
O sistema deve manter o histórico de atividades dos usuários, incluindo a data e o horário em
que eles realizaram o descadastramento de conta