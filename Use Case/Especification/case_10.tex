\section{listar as propostas de aluguel}

\subsection*{Brief Description}
Este caso de uso descreve como os interessados podem visualizar sua lista de
propostas de aluguel cadastradas

\subsection*{Actors}

\begin{itemize}
    \item \textbf{Interessados}
    % \item \textbf{Proprietários}
\end{itemize}

\subsection*{Preconditions}
O interessado deve ter uma conta cadastrada no sistema.

\subsection*{Basic Flow of Events}

\begin{enumerate}
    \item  O interessado seleciona a opção relacionada à visualizar sua lista de propostas de aluguel.
    \item O sistema mostra uma lista com todas as propostas de aluguel cadastradas pelo
    interessado.
    \item O usuário pode selecionar as opções de visualizar os dados de alguma proposta ou a
    opção de descadastrar a determinada proposta.
\end{enumerate}

\subsection*{Alternative Flows}

\textbf{Nenhuma proposta cadastrada}


Caso não haja nenhuma proposta cadastrada, o sistema informa ao interessado que ele não
possui nenhuma proposta para ser visualizada.

\subsection*{Post-conditions}

\textbf{Falha ao visualizar lista de propostas de aluguel}

Caso algum erro inesperado aconteça, o sistema reinicia a operação de visualizar a lista de propostas de aluguel

\subsection*{Special Requirements} 
O sistema deve manter o histórico de atividades dos usuários, incluindo as últimas propostas de
aluguel visualizadas.
