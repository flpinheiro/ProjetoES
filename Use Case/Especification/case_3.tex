\section{apresentar dados de imóvel}

\subsection*{Brief Description}
Este caso de uso descreve como os interessados veem os dados a respeito do imóvel desejado.

\subsection*{Actors}

\begin{itemize}
    \item \textbf{Interessados}
    \item \textbf{Proprietários}
\end{itemize}

\subsection*{Preconditions}
 
Devem ter sido encontrados imóveis disponíveis de acordo com os requisitos desejados pelo
Interessado.

\subsection*{Basic Flow of Events}

\begin{enumerate}
    \item O Interessado seleciona o imóvel desejado entre os disponíveis na lista de imóveis encontrados.
    \item O sistema fornece os dados que foram fornecidos pelo Proprietário do imóvel.
    \item O Interessado confirma que tem interesse em alugar o determinado imóvel.
    \item O sistema solicita que o Interessado esteja autenticado.
    \item  O Interessado se autentica e cadastra uma proposta de aluguel.
\end{enumerate}

\subsection*{Alternative Flows}

\textbf{O Interessado não tem interesse no imóvel}

O sistema permite que o Interessado retorne à página inicial.
\textbf{O Interessado não é autenticado.}
O sistema impede que o Interessado cadastre uma proposta de aluguel até que ele realize a
autenticação.

\subsection*{Post-conditions}

\textbf{Proposta de aluguel cadastrada}
A nova proposta de aluguel do Interessado é cadastrada e pode ser verificada pelo proprietário do
imóvel.


\subsection*{Special Requirements} 
O sistema deve manter o histórico de imóveis visualizados pelos Interessados.