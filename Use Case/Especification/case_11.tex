\section{apresentar dados de proposta de aluguel}

\subsection*{Brief Description}
Este caso de uso descreve como os interessados podem visualizar os dados de
suas propostas de aluguel cadastradas.

\subsection*{Actors}

\begin{itemize}
    \item \textbf{Interessados}
    % \item \textbf{Proprietários}
\end{itemize}

\subsection*{Preconditions}
O interessado deve ter alguma proposta de aluguel cadastrada no sistema.

\subsection*{Basic Flow of Events}

\begin{enumerate}
    \item O interessado seleciona a opção relacionada à visualizar sua lista de propostas de aluguel.
    \item O sistema mostra uma lista com todas as propostas de aluguel cadastradas pelo
    interessado.
    \item  O usuário seleciona a proposta ao qual deseja ver os dados.
    \item  O sistema mostra os seguintes dados sobre a respectiva proposta selecionada: valor à
    ser pago, período do aluguel e número de hóspedes.
\end{enumerate}

\subsection*{Alternative Flows}

Sem alternative flows.

\subsection*{Post-conditions}

\textbf{Falha ao visualizar os dados da proposta de aluguel}

Caso algum erro inesperado aconteça, o sistema reinicia a operação de visualizar os dados da
proposta de aluguel.

\subsection*{Special Requirements} 
O sistema deve manter o histórico de atividades dos usuários, incluindo as últimas propostas de
aluguel visualizadas.