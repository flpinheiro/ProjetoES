\section{Descadastrar Imóvel}

\subsection*{Brief Description}
Este caso de uso descreve como os proprietários podem descadastrar imóveis que foram cadastrados.

\subsection*{Actors}

\begin{itemize}
    % \item \textbf{Interessados}
    \item \textbf{Proprietários} 
\end{itemize}

\subsection*{Preconditions}
O proprietário deve ter imóveis cadastrados.

\subsection*{Basic Flow of Events}

\begin{enumerate}
    \item proprietário acessa a lista de imóveis cadastrados da sua conta.
    \item Proprietário escolhe o imóvel que deseja descadastrar e seleciona a opção referente ao descadastramento.
    \item O sistema pede para que o proprietário confirme sua senha para confirmar sua identidade. 
    \item O proprietário informa sua senha.
    \item O sistema descadastra o respectivo imóvel.
\end{enumerate}

\subsection*{Alternative Flows}

\textbf{Senha inválida}

O sistema informa ao usuário que a senha fornecida não corresponde com a que foi criada no cadastro da conta, permitindo que ele informe a senha novamente.

\subsection*{Post-conditions}

\textbf{Imóvel descadastrado com sucesso}

Se a ação ocorrer corretamente, o sistema informa ao usuário que o imóvel foi descadastrado e permite que ele retorne à lista de imóveis restantes.

\textbf{Falha ao descadastrar imóvel}

Caso algum erro inesperado aconteça o sistema reinicia a operação de descadastro de imóveis.

\subsection*{Special Requirements} 
O sistema deve manter o histórico de atividades dos usuários, incluindo o dia e o horário em que o descadastro foi realizado
