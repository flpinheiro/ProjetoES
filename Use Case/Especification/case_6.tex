\section{Autenticar usuário}

\subsection*{Brief Description}
Este caso descreve como o proprietário deve prover os dados de seus imóveis que serão
anunciados.

\subsection*{Actors}

\begin{itemize}
    % \item \textbf{Interessados}
    \item \textbf{Proprietários}
\end{itemize}

\subsection*{Preconditions}
O usuário deve ter a conta autenticada no sistema.

\subsection*{Basic Flow of Events}

\begin{enumerate}
    \item O usuário seleciona a opção referente a cadastrar um imóvel.
    \item O sistema pede para que sejam fornecidos os seguintes dados do imóvel: classe do imóvel
    (apartamento, casa ou quarto), descrição, endereço, número máximo de hóspedes, data inicial do
    período de disponibilidade, data final do período de disponibilidade e valor de diária mínimo.
\end{enumerate}

\subsection*{Alternative Flows}

\textbf{usuário não autenticado}

O sistema oferece as opções de redirecionar o usuário para a página de autenticação ou para a
página inicial.

\textbf{Endereço inválido}

O sistema informa que o endereço fornecido não é válido e pede para que o mesmo seja corrigido.

\textbf{Limite de imóveis cadastrados}

Cada usuário pode cadastrar apenas 5 imóveis, caso o usuário tente cadastrar um sexto imóvel o
sistema informa que não é possível completar a ação e retorna o usuário para a página inicial.

\subsection*{Post-conditions}

\textbf{Imóvel cadastrado com sucesso}

Se o usuário consegue cadastrar o imóvel com sucesso o sistema o-redireciona para a página de
consulta deste imóvel.

\textbf{Falha ao cadastrar imóvel}

Caso algum erro inesperado aconteça o sistema reinicia a operação de cadastro de imóveis.


\subsection*{Special Requirements} 
O sistema deve manter o histórico de atividades dos usuários, incluindo o dia e o horário em que
acessaram o sistema.
