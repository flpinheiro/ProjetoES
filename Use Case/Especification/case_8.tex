\section{editar dados de imóvel}

\subsection*{Brief Description}
Este caso de uso descreve como os proprietários podem alterar os dados dos imóveis
cadastrados.

\subsection*{Actors}

\begin{itemize}
    % \item \textbf{Interessados}
    \item \textbf{Proprietários}
\end{itemize}

\subsection*{Preconditions}
O proprietário deve ter imóveis cadastrados.

\subsection*{Basic Flow of Events}

\begin{enumerate}
    \item  O proprietário acessa a lista de imóveis cadastrados da sua conta.
    \item O proprietário escolhe o imóvel do qual deseja editar os dados e seleciona a opção referente à
    edição de dados
    \item O proprietário escolhe quais dados deseja alterar (classe do imóvel, descrição, endereço, número
    máximo de hóspedes, data inicial do período de disponibilidade, data final do período de
    disponibilidade e valor mínimo de diária).
    \item O sistema pede para que o proprietário confirme sua senha para confirmar sua identidade.
    \item O proprietário informa sua senha.
    \item O sistema atualiza os novos dados do respectivo imóvel.
\end{enumerate}

\subsection*{Alternative Flows}

\textbf{Dados inválidos}

Caso algum dos novos dados informados não seja válido, o sistema informa o dado que está
incorreto e pede para que o proprietário o-corrija.

\textbf{Senha inválida}

O sistema informa ao usuário que a senha fornecida não corresponde com a que foi criada no
cadastro da conta, permitindo que ele informe a senha novamente.

\subsection*{Post-conditions}

\textbf{Dados editados com sucesso}

Se a ação ocorrer corretamente, o sistema informa ao proprietário que os dados foram
corretamente alterados e permite que ele retorne à lista de imóveis.

\textbf{Falha ao editar dados}

Caso algum erro inesperado aconteça o sistema informa ao proprietário que houve um erro e
reinicia a operação de edição de dados.

\subsection*{Special Requirements} 
O sistema deve manter o histórico de atividades dos usuários, incluindo o dia e o horário em que
os dados foram alterados.