\section{editar dados da conta}

\subsection*{Brief Description}
Este caso descreve como os usuários podem alterar os dados da conta autenticada.

\subsection*{Actors}

\begin{itemize}
    \item \textbf{Interessados}
    \item \textbf{Proprietários}
\end{itemize}

\subsection*{Preconditions}
O usuário deve ter a conta autenticada no sistema.

\subsection*{Basic Flow of Events}

\begin{enumerate}
    \item  O usuário seleciona a opção referente a editar os dados da sua conta.
    \item O sistema oferece as opções de edição para o nome, telefone, senha e endereço de e-mail.
    \item O usuário escolhe a opção que deseja editar.
    \item O sistema pede para que o usuário digite a alteração que deseja fazer.
    \item O sistema envia um e-mail para o endereço de e-mail cadastrado até então pelo usuário para
    que ele confirme a operação.
    \item O usuário realiza a confirmação por meio de seu endereço de e-mail.
    \item O sistema atualiza e registra os novos dados do usuário.
\end{enumerate}

\subsection*{Alternative Flows}

\textbf{E-mail inválido}

Caso o novo endereço de e-mail adicionado pelo usuário não seja válido, o sistema informa o
erro ao usuário e pede para que ele digite um outro endereço de e-mail.


\subsection*{Post-conditions}

\textbf{Alterações realizadas com sucesso}
Se o usuário consegue editar os dados da sua conta com sucesso o sistema o-informa que os
dados foram alterados e o-redireciona para a página inicial.


\textbf{Falha ao editar os dados}
Caso algum erro inesperado aconteça, o sistema reinicia a operação de edição de dados.

\subsection*{Special Requirements} 
O sistema deve manter o histórico de atividades dos usuários, incluindo a data e o horário em
que eles realizaram as alterações dos dados da conta