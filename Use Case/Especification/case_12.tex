\section{descadastrar proposta de aluguel}

\subsection*{Brief Description}
Este caso de uso descreve como os interessados podem descadastrar suas
propostas de aluguel cadastradas.

\subsection*{Actors}

\begin{itemize}
    \item \textbf{Interessados}
    % \item \textbf{Proprietários}
\end{itemize}

\subsection*{Preconditions}
O interessado deve ter alguma proposta de aluguel cadastrada no sistema.

\subsection*{Basic Flow of Events}

\begin{enumerate}
    \item  O interessado seleciona a opção relacionada à visualizar sua lista de propostas de aluguel.
    \item O sistema mostra uma lista com todas as propostas de aluguel cadastradas pelo
    interessado.
    \item O interessado seleciona a proposta ao qual deseja descadastrar.
    \item O sistema requisita que o interessado forneça a senha de sua conta cadastrada para
    confirmar sua identidade.
    \item O sistema confirma a senha do interessado e realiza o descadastro da respectiva
    proposta.
\end{enumerate}

\subsection*{Alternative Flows}

\textbf{Senha inválida}

Caso a senha fornecida seja inválida, o sistema pede para que o interessado forneça a senha
novamente.


\subsection*{Post-conditions}

\textbf{Proposta de aluguel descadastrada com sucesso}

Caso o procedimento ocorra corretamente, o sistema informa ao interessado que a proposta de
aluguel foi descadastrada e permite que ele retorne à sua lista de propostas ainda cadastradas.

\textbf{Erro ao descadastrar proposta de aluguel}

Caso algum erro inesperado aconteça, o sistema reinicia a operação de descadastrar propostas
de aluguel.

\subsection*{Special Requirements} 
O sistema deve manter o histórico de atividades dos usuários, incluindo a data e o horário em que
os interessados descadastraram as suas propostas.